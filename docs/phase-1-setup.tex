\documentclass[11pt,a4paper]{article}

\usepackage[margin=1in]{geometry}
\usepackage{setspace}
\usepackage{hyperref}
\usepackage{titlesec}

\titleformat{\section}{\large\bfseries}{\thesection.}{0.5em}{}
\titleformat{\subsection}{\normalsize\bfseries}{\thesubsection}{0.5em}{}

\title{\textbf{Phase 1 Documentation}\\
Workflow \& Approval Management System (WAMS)}
\author{Backend Project Setup and Bootstrapping}
\date{}

\begin{document}
\maketitle
\vspace{-1em}

\onehalfspacing

\section{Objective of Phase 1}

The objective of Phase 1 was to establish a clean, professional, and interview-ready backend foundation for the \textbf{Workflow \& Approval Management System (WAMS)} using Java and Spring Boot.

This phase intentionally focused on:
\begin{itemize}
  \item Correct environment setup
  \item Understanding Spring Boot bootstrapping
  \item Establishing project structure
  \item Version control discipline
\end{itemize}

No business logic or database integration was implemented in this phase.

---

\section{Technology Stack}

\begin{itemize}
  \item \textbf{Language:} Java 17
  \item \textbf{Framework:} Spring Boot 3.5.10
  \item \textbf{Build Tool:} Maven (Maven Wrapper)
  \item \textbf{Version Control:} Git
  \item \textbf{Repository Hosting:} GitHub
  \item \textbf{Architecture Style:} Layered Monolith (initial)
\end{itemize}

---

\section{System Environment Setup}

\subsection{Java Installation}

Java Development Kit (JDK) 17 was installed and configured on the system.  
The following environment variables were verified:

\begin{itemize}
  \item \texttt{JAVA\_HOME} pointing to the JDK root directory
  \item \texttt{PATH} including \texttt{\%JAVA\_HOME\%/bin}
\end{itemize}

Verification commands:
\begin{verbatim}
java -version
javac -version
\end{verbatim}

---

\subsection{Maven Usage}

Instead of relying on a system-installed Maven, the project uses the \textbf{Maven Wrapper} (\texttt{mvnw}), ensuring:
\begin{itemize}
  \item Build reproducibility
  \item Tooling consistency across machines
\end{itemize}

---

\section{Spring Boot Project Initialization}

The project was initialized using Spring Initializr with the following configuration:

\begin{itemize}
  \item Project Type: Maven
  \item Packaging: Jar
  \item Java Version: 17
  \item Dependencies:
    \begin{itemize}
      \item Spring Web
      \item Spring Data JPA
      \item MySQL Driver
    \end{itemize}
\end{itemize}

At this stage, database connectivity was intentionally deferred.

---

\section{Application Bootstrapping}

The application entry point is defined by the main class:

\begin{verbatim}
@SpringBootApplication
public class ArwmsApplication {
    public static void main(String[] args) {
        SpringApplication.run(ArwmsApplication.class, args);
    }
}
\end{verbatim}

This class is responsible for:
\begin{itemize}
  \item Component scanning
  \item Auto-configuration
  \item Application context initialization
\end{itemize}

---

\section{Handling DataSource Auto-Configuration}

Since JPA and a database driver were added as dependencies, Spring Boot attempted to auto-configure a DataSource.

As database integration was not part of Phase 1, auto-configuration was explicitly disabled using:

\begin{verbatim}
spring.autoconfigure.exclude=
org.springframework.boot.autoconfigure.jdbc.DataSourceAutoConfiguration
\end{verbatim}

This ensured:
\begin{itemize}
  \item Clean application startup
  \item Intentional control over when database configuration is introduced
\end{itemize}

---

\section{Project Naming and Domain Clarification}

Initially, the project followed an academic-specific naming convention.  
This was later generalized to an industry-oriented system:

\begin{itemize}
  \item \textbf{Final Project Name:} Workflow \& Approval Management System (WAMS)
  \item \textbf{Repository Name:} \texttt{wams-backend}
\end{itemize}

The internal package naming was intentionally left unchanged at this stage to avoid premature refactoring. Domain-specific naming will be introduced alongside entity design in later phases.

---

\section{Version Control Setup}

\subsection{Local Git Initialization}

Git was initialized locally, and a clean commit history was established from the first successful boot.

\begin{verbatim}
git init
git add .
git commit -m "Initial Spring Boot setup and successful startup"
\end{verbatim}

---

\subsection{Branch Management}

The default branch was renamed to align with modern GitHub conventions:

\begin{verbatim}
git branch -M main
\end{verbatim}

---

\subsection{GitHub Integration}

The local repository was connected to GitHub and pushed:

\begin{verbatim}
git remote add origin <repository-url>
git push -u origin main
\end{verbatim}

This established GitHub as the single source of truth for the project.

---

\section{Outcome of Phase 1}

At the end of Phase 1, the project achieved the following:

\begin{itemize}
  \item A clean, running Spring Boot application
  \item Verified Java and Maven environment
  \item Intentional control over auto-configuration
  \item Professional Git and GitHub workflow
  \item A stable foundation for domain-driven backend development
\end{itemize}

---

\section{Next Phase}

Phase 2 will focus on:
\begin{itemize}
  \item REST API fundamentals
  \item Controller layer design
  \item Request–response flow
  \item Establishing the first externally visible API of WAMS
\end{itemize}

\end{document}
